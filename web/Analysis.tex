\nwfilename{web/Analysis.tex}
\chapter{Analysis}
\label{chapter-analysis}

This chapter presents the ``Software Requirements Specification'' (SRS) largely
based on ISO/IEC/IEEE 29148:2011(E) ``Systems and software engineering -- Life
cycle processes -- Requirements engineering''.

\section{Introduction}

This section gives a scope description and overview of everything needed to
design and implement \LibName (library) by taking the requirements listed in
this section into account.

\subsection{Purpose}

The SRS analyses the problem---molecular integral evaluation---that \LibName
aims to solve, and lays out the requirements that \LibName should supply or
should not supply.

The SRS is primarily intended to prepare a reference for developing \LibName
(Version \LibVersion). It can also be proposed to \LibName users for their
suggestions and comments.

\subsection{Scope}

%This section should
%a) Identify the software product(s) to be produced by name (e.g., Host DBMS, Report Generator, etc.);
%b) Explain what the software product(s) will, and, if necessary, will not do;
%c) Describe the application of the software being specified, including relevant benefits, objectives, and
%goals;
%d) Be consistent with similar statements in higher-level specifications (e.g., the system requirements
%specification), if they exist.

\subsection{Product Overview}

\subsubsection{Product Perspective}

\subsubsection{Product Functions}

\subsubsection{User Characteristics}

\subsubsection{Limitations}

\subsection{Definitions}

\section{References}

\section{Specific Requirements}

%This section of the SRS should contain all of the software requirements to a
%level of detail sufficient to enable designers to design a system to satisfy
%those requirements, and testers to test that the system satisfies those
%requirements. Throughout this section, every stated requirement should be
%externally perceivable by users, operators, or other external systems. These
%requirements should include at a minimum a description of every input
%(stimulus) into the system, every output (response) from the system, and all
%functions performed by the system in response to an input or in support of an
%output. As this is often the largest and most important part of the SRS, the
%following principles apply:
%a) Specific requirements should be stated in conformance with all the characteristics described in 4.3.
%b) Specific requirements should be cross-referenced to earlier documents that relate.
%c) All requirements should be uniquely identifiable.
%d) Careful attention should be given to organizing the requirements to maximize readability.

\subsection{External Interfaces}

\subsection{Functions}

\subsection{Usability Requirements}

\subsection{Performance Requirements}

\subsection{Logical Database Requirements}

\subsection{Design Constraints}

\subsection{Software System Attributes}

\subsection{Supporting Information}

\section{Verification}

\section{Appendices}

%When appendixes are included, the SRS should explicitly state whether or not
%the appendixes are to be considered part of the requirements.

\subsection{Assumptions and Dependencies}

\subsection{Acronyms and Abbreviations}

\subsection{Theoretical Background\index{Theoretical background}}
\label{subsection-theory}

For the time being, \LibName has implemented the density matrix-based
quasienergy formulation of the Kohn--Sham density functional response theory
using perturbation- and time-dependent basis
sets~\cite{Thorvaldsen-JCP-129-214108,Bast-PCCP-13-2627}.

The density matrix-based quasienergy formulation actually works for different
levels of theory, i.e., one-, two- and four-component levels. A relativistic
implementation can be found in Ref.~\cite{Bast-CP-356-177}.

\LibName uses the recursive programming techniques~\cite{Ringholm-JCC-35-622}
to compute different molecular properties order by order. The recursive
programming techniques can also be used for calculations of residues, the
implementation of the first order residues can be found in
Ref.~\cite{Friese-JCTC-11-1129}.

\subsection{Open-Ended Response Theory\index{Open-ended response theory}}
\label{subsection-open-ended}

The name \LibName stands for \textbf{open-ended response theory}, that is,
the library is:
\begin{enumerate}
  \item open-ended for different levels of theory, i.e., one-, two- and
    four-component levels;
  \item open-ended for different wave functions, e.g., atomic-orbital~(AO)
    based density matrix, molecular orbital~(MO) cofficients and
    coupled cluster~(CC);
  \item open-ended for different kinds of perturbations; and
  \item open-ended for different host programs.
\end{enumerate}

As aformentioned, \LibName has for the time being implemented the AO based
density matrix response theory (source codes in
\texttt{src/ao\_dens})\footnote{The codes in \texttt{src/ao\_dens} are written
in Fortran, but \LibName APIs are implemented using C language. Therefore,
adapter codes between them are implemented in \texttt{src/ao\_dens/adapter},
for \LibName APIs calling the codes of AO based density matrix response theory,
also for the AO based density matrix response theory codes calling the callback
functions (as function pointers saved by \LibName APIs).}, and it works for
one-, two- and four-component levels by simply setting the appropriate
Hamiltonian. We are now planning to implement the MO and CC based response
theories.

To make \LibName work for any perturbation, we will implement the so called
\textbf{perturbation free scheme}, see Section~\ref{subsection-analysis-perturbation}.

In order to make it easy for implementing \LibName into different host programs
(written in different programming languages), we agree to use the
\textbf{callback function scheme}\index{Callback function scheme} in \LibName
in the 2015 Skibotn meeting. The callback functions are specified by host
programs by calling the \LibName application program interface (APIs, both C
and Fortran implemented) during run time, and will be used by \LibName during
calculations, to get contributions from electronic and nuclear Hamiltonian, and
to get response parameters from solving the linear response equation.

Another important issue affects the implementation of \LibName into different
host programs is the matrix and its different operations that \LibName
extensively depends on. Different host programs can have different types of
matrices (dense and sparse, sequential and parallel) and written by different
programming languages (e.g. C and Fortran).

To best utilize the host program's developed matrix routines (if there is), and
also to remove this complexity of matrix problem from \LibName, we also agree
to build \LibName on top of the
\href{https://gitlab.com/bingao/qcmatrix}{\textbf{\textsc{QcMatrix} library}}\index{\textsc{QcMatrix} library}
in the 2015 Skibotn meeting. This matrix library works as an adapter between
\LibName and different matrix routines (implemented in different host programs)
that can be written in C and Fortran\footnote{If there is no matrix routines
implemented in a host program, it can fully use the QcMatrix library that will
invoke BLAS and LAPACK libraries for matrix operations.}.

\subsection{\LibName Framework\index{\LibName framework}}

Therefore, a full picture of \LibName used in a C host program can be (the
description of \LibName Fortran APIs can be found in
Section~\ref{section-OpenRSP-Fortran}\index{\LibName Fortran APIs}):

\begin{figure}[hbt]
  \centering
  \scalebox{0.7}{\documentclass{standalone}

\usepackage[latin1]{inputenc}
\usepackage{amsmath}
\usepackage{amssymb}
\usepackage{amsthm}

\usepackage{tikz}
\usetikzlibrary{arrows,calc}

%% generates a tightly fitting border around the work
%\usepackage[active,tightpage]{preview}
%\PreviewEnvironment{tikzpicture}
%\setlength\PreviewBorder{0.5mm}
%%\renewcommand\PreviewBbAdjust{-\PreviewBorder 1mm -1.15mm -0.85mm}

\usepackage{color}

%\pagestyle{empty}

\begin{document}

\begin{tikzpicture}[thick]
  \node[color=black, rectangle, draw, text badly centered, rounded corners, %
        minimum height=20] (OpenRSP-response) {OpenRSP response};
  \node[color=black, rectangle, draw, text badly centered, rounded corners, %
        minimum height=20, text width=85, left of=OpenRSP-response, node distance=190] %
       (OpenRSP-C-API) {OpenRSP C APIs};
  \node[color=black, rectangle, draw, text badly centered, rounded corners, %
        minimum height=20, text width=95, below of=OpenRSP-response, node distance=100] %
       (OpenRSP-C-support) {OpenRSP C support};
  \node[color=black, dashed, rectangle, draw, text badly centered, rounded corners, %
        minimum height=160, minimum width=400, above of=OpenRSP-C-API, %
        xshift=140, yshift=-70] (OpenRSP-library) %
       {\begin{minipage}[t][5cm]{13.5cm}\textbf{OpenRSP library}\end{minipage}};
%
  \node[color=black, rectangle, draw, text badly centered, rounded corners, %
        minimum height=20, text width=90, left of=OpenRSP-C-API, node distance=140] %
       (HostDriver) {Host program C driver for OpenRSP};
%
  \node[color=red, fill=red!10, rectangle, draw, text badly centered, rounded corners, %
        minimum height=20, text width=70, below of=OpenRSP-C-support, node distance=70, %
        xshift=-150, yshift=0] (Perturbations) {Host program perturbations};
  \node[color=red, fill=red!10, rectangle, draw, text badly centered, rounded corners, %
        minimum height=20, text width=45, below of=OpenRSP-C-support, node distance=70, %
        xshift=-100, yshift=-30] (Overlap) {Overlap integrals};
  \node[color=red, fill=red!10, rectangle, draw, text badly centered, rounded corners, %
        minimum height=20, text width=60, below of=OpenRSP-C-support, node distance=70, %
        xshift=-50, yshift=-60] (OneOper) {One-electron operators};
  \node[color=red, fill=red!10, rectangle, draw, text badly centered, rounded corners, %
        minimum height=20, text width=60, below of=OpenRSP-C-support, node distance=70, %
        xshift=0, yshift=-90] (TwoOper) {Two-electron operators};
  \node[color=red, fill=red!10, rectangle, draw, text badly centered, rounded corners, %
        minimum height=20, text width=70, below of=OpenRSP-C-support, node distance=70, %
        xshift=50, yshift=-60] (XCFun) {XC functionals};
  \node[color=red, fill=red!10, rectangle, draw, text badly centered, rounded corners, %
        minimum height=20, text width=60, below of=OpenRSP-C-support, node distance=70, %
        xshift=100, yshift=-30] (NucHamiltonian) {Nuclear Hamiltonian};
  \node[color=red, fill=red!10, rectangle, draw, text badly centered, rounded corners, %
        minimum height=20, text width=75, below of=OpenRSP-C-support, node distance=70, %
        xshift=150, yshift=0] (Solver) {Linear response equation solver};
  \node[color=red, text badly centered, minimum height=20, text width=100, %
        below of=HostDriver, node distance=240, xshift=180, yshift=0] (Callback) %
       {\textbf{Host program C callback functions for OpenRSP}};
%
  \node[color=black, rectangle, draw, text badly centered, rounded corners, %
        minimum height=20, below of=HostDriver, node distance=240] %
       (QcMatrix) {QcMatrix library};
%
%  \draw [-latex'] (HostDriver) edge node[align=center, midway, yshift=15, text width=120] %
%    {OpenRSP APIs (C or Fortran)} (OpenRSP);
  \draw [-latex'] (OpenRSP-C-API) edge node[align=center, midway, yshift=15, text width=120] %
    {Calculate response functions or residues} (OpenRSP-response);
  \draw [-latex'] (OpenRSP-response) edge node[align=center, midway, xshift=75, text width=160] %
    {Get information of perturbations, contributions from electronic and nuclear Hamiltonian, %
     and response parameters} (OpenRSP-C-support);
  \draw [-latex'] (OpenRSP-C-API) -- %
    node[at end, align=left, xshift=75, yshift=25, text width=140] %
    {Set information of perturbations, electronic and nuclear Hamiltonian, and linear response %
     equation solver} (OpenRSP-C-API |- OpenRSP-C-support) |- (OpenRSP-C-support);
  \draw [-latex'] (OpenRSP-library)--(QcMatrix);
  \draw [-latex'] (HostDriver)--(OpenRSP-C-API);
  \draw [-latex'] (OpenRSP-C-support)--(Perturbations);
  \draw [-latex'] (OpenRSP-C-support)--(Overlap);
  \draw [-latex'] (OpenRSP-C-support)--(OneOper);
  \draw [-latex'] (OpenRSP-C-support)--(TwoOper);
  \draw [-latex'] (OpenRSP-C-support)--(XCFun);
  \draw [-latex'] (OpenRSP-C-support)--(NucHamiltonian);
  \draw [-latex'] (OpenRSP-C-support)--(Solver);
  \draw [-latex'] (HostDriver)--(QcMatrix);
  \draw [-latex'] (Callback)--(QcMatrix);
\end{tikzpicture}

\end{document}
}
  \caption{\LibName used in a C host program.}
  \label{fig-openrsp-framework}
\end{figure}

As shown in Figure~\ref{fig-openrsp-framework}, the \LibName library is divided
into three parts:
\begin{enumerate}
  \item The ``\LibName C APIs'' work mostly between the host program driver
    routine and other parts of the \LibName library, that all the information
    saved in the ``\LibName C support'' will be set up by calling the
    corresponding \LibName C API;
  \item The ``\LibName response'' is the core part in which the performance
    of response theory will be done;
  \item The ``\LibName C support'' saves the information of perturbations,
    electronic and nuclear Hamiltonian and linear response equation solver,
    and will be used by the ``\LibName response'' part during calculating
    response functions and residues.
\end{enumerate}

The ``\LibName response'' was already implemented using Fortran for the AO
based density matrix response theory (source codes in \texttt{src/ao\_dens})
that will not be covered here.

\subsection{Perturbations}
\label{subsection-analysis-perturbation}

For perturbations in \LibName, we introduce the following notations and
convention:
\begin{description}
  \item[Perturbation] is described by a label ($a$), a complex frequency
    ($\omega$) and its order ($n$), and written as $a_{\omega}^{n}$ Any
    two perturbations are different if they have different labels, and/or
    frequencies, and/or orders.
  \item[Perturbation label] is an integer distinguishing one perturbation
    from others; all \textit{different} perturbation labels involved in the
    calculations should be given by calling the application programming
    interface (API) \texttt{OpenRSPSetPerturbations()}; \LibName will stop if
    there is any unspecified perturbation label given afterwards when calling
    the APIs \texttt{OpenRSPGetRSPFun()} or \texttt{OpenRSPGetResidue()}.
  \item[Perturbation order] Each perturbation can acting on molecules once
    or many times, that is the order of the perturbation.
  \item[Perturbation components and their ranks] Each perturbation may have
    different numbers of components for their different orders, the position
    of each component is called its rank.

    For instance, there will usually be $x,y,z$ components for the electric
    dipole perturbation, and their ranks are \texttt{\{0,1,2\}} in zero-based
    numbering, or \texttt{\{1,2,3\}} in one-based numbering.
  \item[Perturbation tuple] An ordered list of perturbation labels, and in
    which we further require that \textit{identical perturbation labels should
    be consecutive}. That means the tuple $(a,b,b,c)$ is allowed, but $(a,b,c,b)$
    is illegal because the identical labels $b$ are not consecutive.

    As a tuple:
    \begin{enumerate}
      \item Multiple instances of the same labels are allowed so that
        $(a,b,b,c)\ne(a,b,c)$, and
      \item The perturbation labels are ordered so that $(a,b,c)\ne(a,c,b)$
        (because their corresponding response functions or residues are in
        different shapes).
    \end{enumerate}
    We will sometimes use an abbreviated form of perturbation tuple as,
    for instance $abc\equiv(a,b,c)$.

    Obviously, a perturbation tuple $+$ its corresponding complex
    frequencies for each perturbation label can be viewed as a set of
    perturbations, in which the number of times a label (with the same
    frequency) appears is the order of the corresponding perturbation.

    For example, a tuple $(a,b,b,c)$ $+$ its complex frequencies
    $(\omega_{a},\omega_{b},\omega_{b},\omega_{c})$ define perturbations
    $a_{\omega_{a}}^{1}$, $b_{\omega_{b}}^{2}$ and $c_{\omega_{c}}^{1}$;
    another tuple $(a,b,b,c)$ $+$ different complex frequencies for labels
    $b$---$(\omega_{a},\omega_{b_{1}},\omega_{b_{2}},\omega_{c})$ define
    different perturbations $a_{\omega_{a}}^{1}$, $b_{\omega_{b_{1}}}^{1}$,
    $b_{\omega_{b_{2}}}^{1}$ and $c_{\omega_{c}}^{1}$.
  \item[Canonical order]~
    \begin{enumerate}
      \item In \LibName, all perturbation tuples are canonically orderd
        according to the argument \texttt{pert\_tuple} in the
        \texttt{OpenRSPGetRSPFun()} or \texttt{OpenRSPGetResidue()}. For
        instance, when a perturbation tuple $(a,b,c)$ given as
        \texttt{pert\_tuple} in the API \texttt{OpenRSPGetRSPFun()},
        \LibName will use such order ($a>b>c$) to arrange all perturbation
        tuples inside and sent to the callback functions.
      \item Moreover, a collection of several perturbation tuples will also
        follow the canonical order. For instance, a collection of all possible
        perturbation tuples of labels $a,b,c$ are $(0,a,b,c,ab,ac,bc,abc)$,
        where $0$ means unperturbed quantities that is always the first one
        in the collection.
    \end{enumerate}
  \item[Perturbation $a$] The first perturbation label in the tuple sent to
    \LibName APIs \texttt{OpenRSPGetRSPFun()} or \texttt{OpenRSPGetResidue()},
    are the perturbation $a$~\cite{Thorvaldsen-JCP-129-214108}.
  \item[Perturbation addressing]~
    \begin{enumerate}
      \item The addressing of perturbation labels in a tuple, as mentioned in
        the term \textbf{Canonical order}, is decided by
        \begin{enumerate}
          \item the argument \texttt{pert\_tuple} sent to the API
            \texttt{OpenRSPGetRSPFun()} or \texttt{OpenRSPGetResidue()}, and
          \item the canonical order that \LibName uses.
        \end{enumerate}
      \item The addressing of components per perturbation (several consecutive
        identical labels with the same complex frequency) are decided by the
        host program, as will be discussed in the following
        \textbf{perturbation free scheme}.
      \item The addressing of a collection of perturbation tuples follows the
        canonical order as aforementioned.
    \end{enumerate}

    Therefore, the shape of response functions or residues is mostly decided
    by the host program. Take $\mathcal{E}^{abbc}$ for example, its shape is
    $(N_{a},N_{bb},N_{c})$, where $N_{a}$ and $N_{c}$ are respectively the
    numbers of components of the first order of the perturbations $a$ and $c$,
    and $N_{bb}$ is the number of components of the second order of the
    perturbation $b$, and
    \begin{enumerate}
      \item In \LibName, we will use notation \texttt{[a][bb][c]} for
        $\mathcal{E}^{abbc}$, where the leftmost index (\texttt{a}) runs
        slowest in memory and the rightmost index (\texttt{c}) runs fastest.
        However, one should be aware that the results are still in a
        one-dimensional array.
      \item If there two different frequencies for the perturbation label
        $b$, \LibName will return \texttt{[a][b1][b2][c]}, where \texttt{b1}
        and \texttt{b2} stand for the components of the first order of the
        perturbation $b$.
      \item The notation for a collection of perturbation tuples (still in a
        one-dimensional array) is
        \texttt{\{1,[a],[b],[c],[a][b],[a][c],[b][c],[a][b][c]\}}
        for $(0,a,b,c,ab,ac,bc,abc)$, where as aforementioned the first one
        is the unperturbed quantities.
    \end{enumerate}
\end{description}

\subsubsection{Perturbation Free Scheme}

Now, let us discuss our \textbf{perturbation free scheme}. As aforementioned,
there could be \textbf{different numbers of components} for different
perturbations. In different host programs, these components could \textbf{be
arranged in different ways}.

For instance, there are 9 components for the second order magnetic derivatives
in a redundant way $xx,xy,xz,yx,yy,yz,zx,zy,zz$, but 6 components in a
non-redundant way $xx,xy,xz,yy,yz,zz$. There are at most four centers in
different integrals, non-zero high order ($\ge 5$) geometric derivatives are
only those with at most four differentiated centers.

To take all the above information into account in \LibName will make it so
complicated and not necessary, because response theory actually does not care
about the detailed knowledge of different perturbations. In particular, when
all the (perturbed) integrals and expectation values are computed by the host
program's callback functions, the detailed information of perturbations:
\begin{enumerate}
  \item the number of components, and
  \item how they are arranged in memory
\end{enumerate}
can be hidden from \LibName.

The former can be easily solved by sending the numbers of components of
different perturbation labels (up to their maximum orders) to the \LibName API
\texttt{OpenRSPSetPerturbations()}.

The latter can be important for \LibName to construct higher-order derivatives
from lower-order ones. We have two cases:
\begin{enumerate}
  \item Higher-order derivatives are taken with respect to different
    perturbations, for instance, $\frac{\partial^{3}}{\partial a\partial b\partial c}$
    are simply the direct product of components of lower-order derivatives
    with respect to each perturbation $\frac{\partial}{\partial a}$,
    $\frac{\partial}{\partial b}$ and $\frac{\partial}{\partial c}$.
  \item Higher-order derivatives are taken with respect to
    \textbf{one perturbation}. Take the second order-derivatives (in the
    redundant format) for example, they can be constructed from the
    first-order ones as,
    \begin{align*}
      x+x\rightarrow xx,\;\; & 0+0\rightarrow 0,\\
      x+y\rightarrow xy,\;\; & 0+1\rightarrow 1,\\
      x+z\rightarrow xz,\;\; & 0+2\rightarrow 2,\\
      y+x\rightarrow yx,\;\; & 1+0\rightarrow 3,\\
      y+y\rightarrow yy,\;\; & 1+1\rightarrow 4,\\
      y+z\rightarrow yz,\;\; & 1+2\rightarrow 5,\\
      z+x\rightarrow zx,\;\; & 2+0\rightarrow 6,\\
      z+y\rightarrow zy,\;\; & 2+1\rightarrow 7,\\
      z+z\rightarrow zz,\;\; & 2+2\rightarrow 8,
    \end{align*}
    where we have ranked different components in zero-based numbering (numbers
    on the right).

    Because the ranks can be different in different host programs, also the
    above mapping relationship between lower- and higher-order derivatives
    (with respect to \textbf{one perturbation}) can be different in different
    host programs.

    We therefore ask for a callback function \texttt{get\_pert\_concatenation()}
    from host programs. This callback function will, from given components of
    a \textbf{concatenated perturbation tuple} (i.e. higher-order derivatives
    with respect to one perturbation), get the ranks of components of the
    \textbf{sub-perturbation tuples with the same perturbation label} (i.e.
    lower-order derivatives with respect to one perturbation).
\end{enumerate}

As such, the numbers of different components of perturbations and their ranks
are totally decided by the host program---that is the
\textbf{perturbation free scheme}.

%\subsubsection{Internal Perturbation Labels}
%
%As mentioned in our notations and convention, perturbations even with the same
%perturbation label, are different if they have different frequencies. Inside
%\LibName, we need to distinguish this kind of different perturbations---same
%perturbation label but different frequencies.
%
%We therefore use internal perturbation labels inside \LibName. To avoid
%introduce new \texttt{struct} for perturbation labels, we use unsigned integers
%for both host program's perturbation labels and \LibName internal perturbation
%labels.
%
%\begin{figure}[hbt]
%  \centering
%  \scalebox{0.9}{\begin{tikzpicture}[thick]
    \node[rectangle, draw](Frequency) at(0,0) {$0000000000001111$};
    \node[left of=Frequency, node distance=80] {Frequency};
    \node[right of=Frequency, node distance=65] {\texttt{<<}16};
    \node[below of=Frequency, node distance=20] {$+$};
    \node[below=of Frequency.west, anchor=west, yshift=-12, %
      rectangle, draw] (HostLabel) {$0000000000000000$ $0000000000000110$};
    \node[left of=HostLabel, node distance=185] %
      {Host program's perturbation label};
    \node[below of=HostLabel, node distance=40, rectangle, draw] %
      (InternalLabel) {$0000000000001111$ $0000000000000110$};
    \node[left of=InternalLabel, node distance=168] %
      {Internal perturbatoin label};
    \draw[-latex'](HostLabel)--(InternalLabel);
%
    \node[below of=InternalLabel, node distance=20] {\texttt{\&}};
    \node[below of=InternalLabel, node distance=40, rectangle, draw] %
      (Filter) {$0000000000000000$ $1111111111111111$};
    \node[left of=Filter, node distance=115] {$2^{16}-1$};
    \draw[-latex'](Filter.east)--++(1,0)node{~}|-(HostLabel.east);
\end{tikzpicture}
}
%  \caption{Perturbation labels in \LibName.}
%  \label{fig-perturbation-label}
%\end{figure}
%
%As illustrated in Figure~\ref{fig-perturbation-label}, if one uses lower $16$
%bits for a host program's label and higher $16$ bits for a frequency in the
%$32$-bit unsigned integers, and if both the host program's label and the
%frequency are marked from $0$ up to $2^{16}-1=65535$, \LibName can thus treat
%$2^{16}=65536$ different perturbation labels and different frequencies, which
%is enough for the current response theory calculations. If one uses $64$-bit
%unsigned integers, \LibName can then treat more perturbation labels and
%frequencies.

%CCCCCCCCCCCCCCCCCCCCCCCCCCCCCCCCCCCCCCCCCCCCCCCCCCCCCCCCCCCCCCCCCCCCCC

%\subsection{Internal Perturbation Labels}
%\label{subsection-OpenRSP-intern-pert}
%
%As described in Figure~\ref{fig-perturbation-label}, we will use an unsigned
%integer to represent perturbation labels for both the host program and the
%internal use of \LibName. The type of this unsigned integer
%[[QcPertInt]]\index{[[QcPertInt]]} is defined as follows:
%<<RSPPertBasicTypes>>=
%/* <macrodef name='OPENRSP_PERT_SHORT_INT'>
%     Represent perturbation labels using unsigned short integers
%   </macrodef> */
%#if defined(OPENRSP_PERT_SHORT_INT)
%/* <datatype name='QcPertInt'>
%     Data type of integers to represent perturbation labels
%   </datatype>
%   <constant name='QCPERTINT_MAX'>
%     Maximal value of an object of the <QcPertInt> type
%   </constant>
%   <constant name='QCPERTINT_FMT'>
%     Format string of <QcPertInt> type
%   </constant> */
%typedef unsigned short QcPertInt;
%#define QCPERTINT_MAX USHRT_MAX
%#define QCPERTINT_FMT "hu"
%/* <macrodef name='OPENRSP_PERT_INT'>
%     Represent perturbation labels using unsigned integers
%   </macrodef> */
%#elif defined(OPENRSP_PERT_INT)
%typedef unsigned int QcPertInt;
%#define QCPERTINT_MAX UINT_MAX
%#define QCPERTINT_FMT "u"
%#else
%typedef unsigned long QcPertInt;
%#define QCPERTINT_MAX ULONG_MAX
%#define QCPERTINT_FMT "lu"
%#endif
%@ Here we allow users to choose either
%[[unsigned short]]\index{[[OPENRSP_PERT_SHORT_INT]]},
%[[unsigned int]]\index{[[OPENRSP_PERT_INT]]}, or [[unsigned long]] for the
%type [[QcPertInt]]. We also define a constant
%[[QCPERTINT_MAX]]\index{[[QCPERTINT_MAX]]} for the maximal value of an object
%of the [[QcPertInt]] type, and a format string
%([[QCPERTINT_FMT]]\index{[[QCPERTINT_FMT]]}) of the [[QcPertInt]] type.
%
%We futher allow users to set the number of bits in an object of [[QcPertInt]]
%for representing the host program's perturbation labels (see
%Figure~\ref{fig-perturbation-label}). This can be done by changing the constant
%[[OPENRSP_PERT_LABEL_BIT]]\index{[[OPENRSP_PERT_LABEL_BIT]]} during building:
%<<RSPPertBasicTypes>>=
%/* <macrodef name='OPENRSP_PERT_LABEL_BIT'>
%     Set <OPENRSP_PERT_LABEL_BIT>
%   </macrodef>
%   <constant name='OPENRSP_PERT_LABEL_BIT'>
%     Number of bits in an object of <QcPertInt> type for a perturbation label
%   </constant> */
%#if !defined(OPENRSP_PERT_LABEL_BIT)
%#define OPENRSP_PERT_LABEL_BIT 10
%#endif
%@ and from which, and from the knowledge of [[QCPERTINT_MAX]] we can compute
%[[OPENRSP_PERT_LABEL_MAX]]\index{[[OPENRSP_PERT_LABEL_MAX]]} and
%[[OPENRSP_NUM_FREQ_MAX]]\index{[[OPENRSP_NUM_FREQ_MAX]]}, which are the maximal
%values of perturbation labels and the number of frequencies allowed:
%<<RSPPertBasicTypes>>=
%/* <constant name='OPENRSP_PERT_LABEL_MAX'>
%     Maximal value for perturbation labels
%   </constant>
%   <constant name='OPENRSP_NUM_FREQ_MAX'>
%     Maximal value for number of frequencies
%   </constant> */
%extern const QcPertInt OPENRSP_PERT_LABEL_MAX;
%extern const QcPertInt OPENRSP_NUM_FREQ_MAX;
%@ Here, to avoid multiple inclusions of the header file that will lead to
%multiple definitions, we have the following implementation file for the
%[[OPENRSP_PERT_LABEL_MAX]] and [[OPENRSP_NUM_FREQ_MAX]]:
%<<RSPPertLabel.c>>=
%/*
%  <<OpenRSPLicense>>
%*/
%
%#include "RSPPerturbation.h"
%
%/* see https://scaryreasoner.wordpress.com/2009/02/28/checking-sizeof-at-compile-time
%   accessing date Oct. 6, 2015 */
%#define QC_BUILD_BUG_ON(condition) ((void)sizeof(char[1 - 2*!!(condition)]))
%void RSPPertCheckLabelBit()
%{
%    QC_BUILD_BUG_ON(sizeof(QCPERTINT_MAX)*CHAR_BIT<=OPENRSP_PERT_LABEL_BIT);
%}
%
%const QcPertInt OPENRSP_PERT_LABEL_MAX = (1<<OPENRSP_PERT_LABEL_BIT)-1;
%const QcPertInt OPENRSP_NUM_FREQ_MAX =
%                (QCPERTINT_MAX-OPENRSP_PERT_LABEL_MAX)>>OPENRSP_PERT_LABEL_BIT;
%@ The function [[RSPPertCheckLabelBit()]] ensures that
%[[OPENRSP_PERT_LABEL_BIT]] is not too large and there are still bits left for
%the number of frequencies. One will have building error when compiling the
%function [[RSPPertCheckLabelBit()]] if [[OPENRSP_PERT_LABEL_BIT]] is too large.
%
%However, the function [[RSPPertCheckLabelBit()]] can not guarantee the above
%setting ([[QcPertInt]] type and [[OPENRSP_PERT_LABEL_BIT]]) is enough for
%holding the host program's perturbation labels and the number of frequencies.
%This will be checked against [[OPENRSP_PERT_LABEL_MAX]] and
%[[OPENRSP_NUM_FREQ_MAX]] by \LibName when (i) setting the host program's
%perturbations, and (ii) calculating response functions or residues.

%\subsection{Conversion of Perturbation Tuples and Labels}
%\label{subsection-OpenRSP-convert-pert}
%
%As described in Figure~\ref{fig-perturbation-label}, we need to convert the
%host program's perturbation labels and frequencies into \LibName internal
%perturbation labels when calculating response functions or residues:
%<<RSPertAPIs>>=
%extern QErrorCode RSPPertTupleHostToInternal(const RSPPert*,
%                                             const QInt,
%                                             const QInt*,
%                                             const QInt,
%                                             const QReal*,
%                                             QInt*,
%                                             QReal*);
%@ This function first looks through a given host program's perturbation tuple
%to find consecutive identical pertubation labels, and then checks how many
%different frequencies associated with the consecutive identical labels:
%<<RSPPerturbation.c>>=
%/* <function name='RSPPertTupleHostToInternal'
%             attr='private'
%             author='Bin Gao'
%             date='2015-10-08'>
%     Convert a host program's perturbation tuple to an internal pertubation tuple
%     <param name='rsp_pert' direction='in'>
%       The context of perturbations
%     </param>
%     <param name='len_tuple' direction='in'>
%       Length of the host program's and the internal perturbation tuples
%     </param>
%     <param name='pert_tuple' direction='in'>
%       The host program's perturbation tuple, in which identical perturbation
%       labels should be consecutive, and the first one is the perturbation $a$
%     </param>
%     <param name='num_freq_configs' direction='in'>
%       Number of different frequency configurations
%     </param>
%     <param name='pert_freqs' direction='in'>
%       Complex frequencies of each perturbation label (except for the
%       perturbation $a$) over all frequency configurations, size is therefore
%       $2\times[(<len_tuple>-1)\times<num_freq_configs>]$, and arranged as
%       <c>[num_freq_configs][len_tuple-1][2]</c> in memory (that is, the real
%       and imaginary parts of each frequency are consecutive in memory)
%     </param>
%     <param name='intern_pert_tuple' direction='out'>
%       The internal perturbation tuple, in which identical perturbation
%       labels are consecutive, and the first one is the perturbation $a$
%     </param>
%     <param name='intern_pert_freqs' direction='out'>
%       Complex frequencies of each perturbation label (including the
%       perturbation $a$) over all frequency configurations, size is therefore
%       $2\times<len_tuple>\times<num_freq_configs>$, and arranged as
%       <c>[num_freq_configs][len_tuple][2]</c> in memory (that is, the real and
%       imaginary parts of each frequency are consecutive in memory)
%     </param>
%     <return>Error information</return>
%   </function> */
%QErrorCode RSPPertTupleHostToInternal(const RSPPert *rsp_pert,
%                                      const QInt len_tuple,
%                                      const QInt *pert_tuple,
%                                      const QInt num_freq_configs,
%                                      const QReal *pert_freqs,
%                                      QInt *intern_pert_tuple,
%                                      QReal *intern_pert_freqs)
%{
%    QInt ipert,jpert;  /* incremental recorders */
%    QInt first_id;     /* first identical pertubation label in the tuple */
%    QInt last_id;      /* last identical pertubation label in the tuple */
%    QBool non_id;      /* indicates if non-identical label found */
%
%    /* we first try to find consecutive identical pertubation labels */
%    first_id = 0;
%    non_id = QFALSE;
%    for (ipert=first_id; ipert<len_tuple-1; ipert++) {
%        if (pert_tuple[ipert]!=pert_tuple[ipert+1]) {
%            last_id = ipert;
%            non_id = QTRUE;
%            break;
%        }
%    }
%    if (non_id=QTRUE) {
%    }
%    else {
%    }
%    /* loops over all known perturbation labels and checks if they are in the tuple */
%    for (ipert=0; ipert<rsp_pert->num_pert; ipert++) {
%        /* loops over perturbation labels in the tuple */
%        for (ipert=first_id; ipert<len_tuple; ipert++) {
%            /* checks if the given label is known */
%            if (pert_tuple[ipert]==rsp_pert->pert_labels[jpert]) {
%                
%            }
%        }
%    }
%    return QSUCCESS;
%}
%
%@ where we have also checked the numbers of different frequencies against
%[[OPENRSP_NUM_FREQ_MAX]].
%
%After converting into internal perturbation labels, \LibName will usually use
%its internal perturbation tuples (described by the internal perturbation
%labels) and the corresponding complex frequencies in calculations.
%
%However, when making a callback, (i) the complex frequencies will not be
%passed, and (ii) internal perturbation labels will be converted into host
%program's ones. The callback functions can therefore not distinguish different
%perturbations only from the host program's perturbation labels.
%
%For instance, a host program's perturbation tuple $(a,b,b,c)$ and the
%corresponding complex frequencies
%$(\omega_{a},\omega_{b_{1}},\omega_{b_{2}},\omega_{c})$ actually define four
%different perturbations $a_{\omega_{a}}^{1}$, $b_{\omega_{b_{1}}}^{1}$,
%$b_{\omega_{b_{2}}}^{1}$ and $c_{\omega_{c}}^{1}$ instead of three (each
%perturbation label $b$ defines one perturbation).
%
%We therefore need to pass both the host program's perturbation labels
%$(a,b,b,c)$ and their associated orders $(1,1,1,1)$ to callback functions,
%together meaning the passed two perturbation labels $b$ stand for two different
%perturbations, and both of them are the first-order perturbation.
%
%The following function will convert an internal perturbation tuple to the host
%program's pertubation labels and their orders:
%<<RSPertAPIs>>=
%extern QErrorCode RSPPertInternTupleToHostLabelOrder(const QInt,
%                                                     const QInt*,
%                                                     QInt*,
%                                                     QInt*,
%                                                     QInt*);
%@ and here we assume that \textbf{consecutive identical internal perturbation
%labels is for one perturbation}:
%<<RSPPerturbation.c>>=
%/* <function name='RSPPertInternTupleToHostLabelOrder'
%             attr='private'
%             author='Bin Gao'
%             date='2015-10-08'>
%     Convert an internal perturbation tuple to the host program's pertubation
%     labels and their orders
%     <param name='len_tuple' direction='in'>
%       Length of the internal perturbation tuple
%     </param>
%     <param name='intern_pert_tuple' direction='in'>
%       The internal perturbation tuple
%     </param>
%     <param name='num_pert' direction='out'>
%       Number of different perturbations from the internal perturbation tuple
%     </param>
%     <param name='pert_labels' direction='out'>
%       Host program's pertubation labels of the resulted perturbations
%     </param>
%     <param name='pert_orders' direction='out'>
%       Orders of the resulted perturbations
%     </param>
%     <return>Error information</return>
%   </function> */
%QErrorCode RSPPertInternTupleToHostLabelOrder(const QInt len_tuple,
%                                              const QInt *intern_pert_tuple,
%                                              QInt *num_pert,
%                                              QInt *pert_labels,
%                                              QInt *pert_orders)
%      
%{
%    QInt ilab;   /* incremental recorder over perturbation labels */
%    QInt ipert;  /* incremental recorder for different perturbations */
%    for (ilab=0,ipert=0; ilab<len_tuple; ) {
%        pert_labels[ipert] = intern_pert_tuple[ilab];
%        pert_orders[ipert] = 1;
%        /* finds consecutive identical internal perturbation labels */
%        ilab++;
%        for (; ilab<len_tuple; ) {
%            if (pert_labels[ipert]==intern_pert_tuple[ilab]) {
%                pert_orders[ipert]++;
%            }
%            else {
%                break;
%            }
%            ilab++;
%        }
%        /* converts to the host program's label */
%        pert_labels[ipert] &= OPENRSP_PERT_LABEL_MAX;
%        ipert++;
%    }
%    *num_pert = ipert;
%    return QSUCCESS;
%}
%
%@ where the conversion of the internal label to the host program's label is
%achieved by the bitwise AND operation ([[&=]]) with [[OPENRSP_PERT_LABEL_MAX]]
%as illustrated in Figure~\ref{fig-perturbation-label}.

% XC functional callbacks
%
% [0,a,b,c,d,ab,ac,ad,bc,bd,cd]
% ->
% [0,a,b,d,ab,ad,bc,bd]
% if b==c

